%%%%%%%%%%%%%%%%%%%%%%%%%%%%%%%%%%%%%%%%%
% Twenty Seconds Resume/CV
% LaTeX Template
% Version 1.0 (14/7/16)
%
% Original author:
% Carmine Spagnuolo (cspagnuolo@unisa.it) with major modifications by 
% Vel (vel@LaTeXTemplates.com) and Harsh (harsh.gadgil@gmail.com)
%
% License:
% The MIT License (see included LICENSE file)
%
%%%%%%%%%%%%%%%%%%%%%%%%%%%%%%%%%%%%%%%%%

%----------------------------------------------------------------------------------------
%	PACKAGES AND OTHER DOCUMENT CONFIGURATIONS
%----------------------------------------------------------------------------------------

\documentclass[letterpaper]{twentysecondcv} % a4paper for A4

% Command for printing skill overview bubbles
\newcommand\skills{ 
~
	\smartdiagram[bubble diagram]{
        \textbf{Data}\\\textbf{Engineering},
        \textbf{Full Stack}\\\textbf{Dev},
        \textbf{~~~~~~~~OOP~~~~~~~~~},
        \textbf{~~~~~~Machine~~~~~~}\\\textbf{~~Learning~~},
        \textbf{~~~~~DevOps~~~~~}
    }
}

% Programming skill bars
\programming{{Vue.js $\textbullet$ Vuetify.js / 3.5}, {SQL$\textbullet$ HTML$\textbullet$ CSS$\textbullet$JavaScript / 3.5}, {C\# $\textbullet$ Java / 3.5}, {Python $\textbullet$ Django / 3.5}}

% Projects text
\education{
\textbf{Ingeniería Informática} \\
Universidad Tecnológica de la Habana
José Antonio Echeverría \\
2010 - 2016 | Habana, Cuba

\textbf{Título de bachiller}  \\
IPUEC Manuel Ascunce Domenech \\
2006 - 2009 | Habana, Cuba
}

%----------------------------------------------------------------------------------------
%	 INFORMACIÓN PERSONAL
%----------------------------------------------------------------------------------------
% If you don't need one or more of the below, just remove the content leaving the command, e.g. \cvnumberphone{}

\cvname{MADELIN DIAZ} % Your name
\cvjobtitle{ Data Engineer } % Job
% title/career

\cvlinkedin{/in/madelin-díaz-ramos/}
\cvgithub{madelinkind@gmail.com}
\cvnumberphone{(+53) 535 38731} % Phone number
\cvsite{madelinkind.github.io} % Personal website
\cvmail{madelinkind@gmail.com} % Email address

%----------------------------------------------------------------------------------------

\begin{document}

\makeprofile % Print the sidebar
 
%----------------------------------------------------------------------------------------
%	 EXPERIENCIA
%----------------------------------------------------------------------------------------

\section{Experiencia}

\begin{twenty} % Environment for a list with descriptions
\twentyitem
    	{Septiembre 2016 -}
		{Presente}
        {Calidad de software}
        {\href{http://datys.cu/}{DATYS, Habana, Cuba}}
        {}
        {
        {\begin{itemize}
        \item Pruebas de aplicaciones Web y Desktop
        \item Asimilación de la documentación asociada a cada solución
        \item Ejecución y confección de Casos de prueba
        \item Revisión de requisitos funcionales 
        \item Creación y seguimiento de incidencias
        \item Implementación de pruebas unitarias 
        \end{itemize}}
        }
        
	%\twentyitem{<dates>}{<title>}{<location>}{<description>}
\end{twenty}

%----------------------------------------------------------------------------------------
%	 CERTIFICACIÓN Y CURSOS DE FORMACIÓN
%----------------------------------------------------------------------------------------
\section{Certificaciones y cursos de formación}

\begin{twenty} % Environment for a list with descriptions
\twentyitem
    	{2019}
		{}
        {Modelo de la Calidad para el desarrollo de Aplicaciones Informáticas (MCDAI) (DATYS).}
        {}
        {}
        {}
        \\
\twentyitem
    	{2018}
		{}
        {Evaluación de la calidad de productos de software y sistemas (DATYS).}
        {}
        {}
        {}
        \\
\twentyitem
    	{2017}
		{}
        {BigData (DATYS).}
        {}
        {}
        {}
        \\
\twentyitem
    	{2016}
		{}
        {Administración avanzada en GNU/Linux (DATYS).}
        {}
        {}
        {}
        \\
\twentyitem
    	{2015}
		{}
        {Integración de Aplicaciones, Universidad Tecnológica de la Habana José Antonio
Echeverría (CUJAE).}
        {}
        {}
        {}
        \\
\twentyitem
    	{2015}
		{}
        {Análisis y diseño de algoritmos, Universidad Tecnológica de la Habana José Antonio
Echeverría (CUJAE).}
        {}
        {}
        {}
        \\
\twentyitem
    	{2015}
		{}
        {Algoritmos de minería de datos, Universidad Tecnológica de la Habana José Antonio
Echeverría (CUJAE).}
        {}
        {}
        {}
        \\
\twentyitem
    	{2014}
		{}
        {Ciencias del Diseño, Universidad Tecnológica de la Habana José Antonio
Echeverría (CUJAE).}
        {}
        {}
        {}

	%\twentyitem{<dates>}{<title>}{<location>}{<description>}
\end{twenty}
%----------------------------------------------------------------------------------------
%	 IDIOMAS
%----------------------------------------------------------------------------------------
\section{Idiomas}

\begin{twenty} % Environment for a list with descriptions
\twentyitem
    	{Español}
		{}
        {Lengua nativa}
        {}
        {}
        {}
        \\
	\twentyitem
    	{Inglés}
		{}
        {Principiante}
        {}
        {}
        {}


	%\twentyitem{<dates>}{<title>}{<location>}{<description>}
\end{twenty}

\pagebreak

\begin{twenty}
\twentyitem
    	{}
		{}
        {}
        {}
        {}
        {}
        \\
 \end{twenty} 
%----------------------------------------------------------------------------------------
%	 	OTRAS TECNOLOGÍAS
%----------------------------------------------------------------------------------------  
\section{Otras tecnologías}
\begin{twenty}
	
	\twentyitem
    	{IDEs}
		{}
        {}
        {}
        {}
        {
        {\begin{itemize}
        \item Visual Studio, Test Manager(2012, 2013, 2015, 2017)
        \item PyCharm
        \item SeleniumIDE
		\end{itemize}}
        }
        \\
        \twentyitem
    	{DVCS}
		{}
        {}
        {}
        {}
        {
        {\begin{itemize}
        \item Git(GitHub, Git)
        \item TFS
		\end{itemize}}
		}
		\\
        \twentyitem
	    {SO}
		{}
        {}
        {}
        {}
        {
        {\begin{itemize}
        \item Microsoft Windows (XP, 7, 8, 10)
        \item GNU/Linux (Ubuntu)
		\end{itemize}}
		}
		\\
        \twentyitem
	    {UML}
		{}
        {}
        {}
        {}
        {
        {\begin{itemize}
        \item MagicDraw
        \item Bizagi
        \item Mockups
		\end{itemize}}
		}
\end{twenty}
\end{document} 
