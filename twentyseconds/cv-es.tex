%%%%%%%%%%%%%%%%%%%%%%%%%%%%%%%%%%%%%%%%%
% Twenty Seconds Resume/CV
% LaTeX Template
% Version 1.0 (14/7/16)
%
% Original author:
% Carmine Spagnuolo (cspagnuolo@unisa.it) with major modifications by 
% Vel (vel@LaTeXTemplates.com) and Harsh (harsh.gadgil@gmail.com)
%
% License:
% The MIT License (see included LICENSE file)
%
%%%%%%%%%%%%%%%%%%%%%%%%%%%%%%%%%%%%%%%%%

%----------------------------------------------------------------------------------------
%	PACKAGES AND OTHER DOCUMENT CONFIGURATIONS
%----------------------------------------------------------------------------------------

\documentclass[letterpaper]{twentysecondcv} % a4paper for A4

% Command for printing skill overview bubbles
\newcommand\skills{ 
~
	\smartdiagram[bubble diagram]{
        \textbf{Ingeniero de}\\\textbf{Software},
        \textbf{Pruebas}\\\textbf{Unitarias},
        \textbf{~~~~~~~~Q\&A~~~~~~~~~},
        \textbf{~~~~~~Desarrollo~~~~~~}\\\textbf{~~Web~~},
        \textbf{~~~~~POO~~~~~}
    }
}

% Programming skill bars
\programming{{Vue.js $\textbullet$ Vuetify.js / 2.5}, {SQL$\textbullet$ HTML$\textbullet$ CSS$\textbullet$JavaScript / 3}, {C\# $\textbullet$ Java / 3.5}, {Python $\textbullet$ Django / 3}}

% Projects text
\education{
\textbf{Ingeniería Informática} \\
Universidad Tecnológica de la Habana
José Antonio Echeverría (CUJAE) \\
2010 - 2016 | Habana, Cuba

\textbf{Título de bachiller}  \\
IPUEC Manuel Ascunce Domenech \\
2007 - 2010 | Habana, Cuba
}

%----------------------------------------------------------------------------------------
%	 INFORMACIÓN PERSONAL
%----------------------------------------------------------------------------------------
% If you don't need one or more of the below, just remove the content leaving the command, e.g. \cvnumberphone{}

\cvname{MADELIN DIAZ} % Your name
\cvjobtitle{ Ingeniero de Software } % Job
% title/career

\cvlinkedin{/in/madelin-díaz-ramos/}
\cvgithub{github.com/madelinkind}
\cvnumberphone{(+53) 535 38731} % Phone number
\cvsite{madelinkind.github.io} % Personal website
\cvmail{madelinkind@gmail.com} % Email address

%----------------------------------------------------------------------------------------

\begin{document}

\makeprofile % Print the sidebar
 
%----------------------------------------------------------------------------------------
%	 EXPERIENCIA
%----------------------------------------------------------------------------------------

\section{Experiencia}

\begin{twenty} % Environment for a list with descriptions
\twentyitem
    	{09/16 - Ahora}
		{}
        {Calidad de software}
        {\href{http://datys.cu/}{DATYS, Habana, Cuba}}
        {}
        {
        {\begin{itemize}
        \item Asimilación de la documentación asociada a las soluciones de la empresa, asegurando que el software  funcione de acuerdo con los requisitos; trabaje con sus interfaces de la forma esperada, detectando en forma temprana defectos y evitando su propagación y llegada al cliente (interno o externo). Para darle cumplimiento a lo expuesto anteriormente se realizan una serie de actividades mencionadas a continuación:
\item Diseño del plan de prueba para definir los objetivos de las pruebas del sistema.
\item Definir los casos de prueba en base a los requisitos funcionales, no funcionales y técnicos.
\item Test de integración, definir las pruebas de Integración que se realizarán.
\item Ejecutar los casos de prueba
\item Realizar la documentación de las pruebas (evidencia).
\item Registrar los incidentes en la base a los defectos encontrados, así como también realizar su seguimiento para asegurar su adecuada corrección.
\item Analizar y reportar los resultados de las pruebas, estadísticas, etc. 
        \end{itemize}}
        }
        
	%\twentyitem{<dates>}{<title>}{<location>}{<description>}
\end{twenty}

%----------------------------------------------------------------------------------------
%	 CERTIFICACIÓN Y CURSOS DE FORMACIÓN
%----------------------------------------------------------------------------------------
\section{Certificaciones y cursos de formación}

\begin{twenty} % Environment for a list with descriptions
\twentyitem
    	{2019}
		{}
        {Modelo de la Calidad para el desarrollo de Aplicaciones Informáticas (MCDAI) (DATYS).}
        {}
        {}
        {
        {\begin{itemize}
        \item (Taller) Intercambio de experiencias en la aplicación de modelos de calidad, en específico el MCDAI.
		\end{itemize}}
        }
        \\
\twentyitem
    	{2018}
		{}
        {Evaluación de la calidad de productos de software/sistemas (DATYS).}
        {}
        {}
        {
        {\begin{itemize}
        \item (Taller) Ejecución de pruebas a aplicaciones informáticas, dirigidas a evaluar las características de calidad de un software.
		\end{itemize}}
        }
        \\
\twentyitem
    	{2017}
		{}
        {BigData (DATYS).}
        {}
        {}
         {
        {\begin{itemize}
        \item Métodos y herramientas para analizar la información y facilitar la toma de decisiones empresariales.
		\end{itemize}}
        }
        \\
\twentyitem
    	{2016}
		{}
        {Administración avanzada en GNU/Linux (DATYS).}
        {}
        {}
        {
        {\begin{itemize}
        \item Proporcionar desde GNU/Linux los servicios necesarios a diferentes ambientes de usuarios y máquinas.
		\end{itemize}}
        }
        \\
\twentyitem
    	{2015}
		{}
        {Integración de Aplicaciones (CUJAE).}
        {}
        {}
        {
        {\begin{itemize}
        \item Introducción a la integración de sistemas, aplicaciones y base de datos.
		\end{itemize}}
        }
        \\
\twentyitem
    	{2015}
		{}
        {Análisis y diseño de algoritmos (CUJAE).}
        {}
        {}
        {
        {\begin{itemize}
        \item Metodología para la solución de problemas, utilizando la computadora a través del diseño de algoritmos.
		\end{itemize}}
        }
        \\
\twentyitem
    	{2015}
		{}
        {Algoritmos de minería de datos (CUJAE).}
        {}
        {}
        {
        {\begin{itemize}
        \item Introducción a las técnicas básicas de minería de datos.
		\end{itemize}}
        }
        \\
\twentyitem
    	{2014}
		{}
        {Ciencias del Diseño (CUJAE).}
        {}
        {}
        {
        {\begin{itemize}
        \item Fundamentos de la ciencia del diseño.
		\end{itemize}}
        }


	%\twentyitem{<dates>}{<title>}{<location>}{<description>}
\end{twenty}

\pagebreak

\begin{twenty}
\twentyitem
    	{}
		{}
        {}
        {}
        {}
        {}
        \\
 \end{twenty} 

%----------------------------------------------------------------------------------------
%	 IDIOMAS
%----------------------------------------------------------------------------------------
\section{Idiomas}

\begin{twenty} % Environment for a list with descriptions
\twentyitem
    	{Español}
		{}
        {Lengua nativa}
        {}
        {}
        {}
        \\
	\twentyitem
    	{Inglés}
		{}
        {Principiante}
        {}
        {}
        {}


	%\twentyitem{<dates>}{<title>}{<location>}{<description>}
\end{twenty}

%----------------------------------------------------------------------------------------
%	 	OTRAS TECNOLOGÍAS
%----------------------------------------------------------------------------------------  
\section{Tecnologías}
\begin{twenty}
	
	 \twentyitem
    	{Lenguajes}
		{}
        {}
        {}
        {}
        {
        {\begin{itemize}
        \item C\#, Python, Java
		\end{itemize}}
        }
        \\
	\twentyitem
    	{IDEs}
		{}
        {}
        {}
        {}
        {
        {\begin{itemize}
        \item Visual Studio, PyCharm
		\end{itemize}}
        }
        \\
        \twentyitem
    	{DVCS}
		{}
        {}
        {}
        {}
        {
        {\begin{itemize}
        \item Git(GitHub, GitLab), Team Foundation Server(TFS)
		\end{itemize}}
		}
		\\
        \twentyitem
	    {DBMS}
		{}
        {}
        {}
        {}
        {
        {\begin{itemize}
        \item PostgreSQL, SQLite.
		\end{itemize}}
		}
		\\
        \twentyitem
	    {Web}
		{}
        {}
        {}
        {}
        {
        {\begin{itemize}
        \item HTML, CSS, JavaScript, Vue.js, Vuetify.js. Frameworks web como Django.
		\end{itemize}}
		}
		\\
        \twentyitem
	    {SO}
		{}
        {}
        {}
        {}
        {
        {\begin{itemize}
        \item Microsoft Windows (XP, 7, 8, 10), GNU/Linux (Ubuntu)
		\end{itemize}}
		}
		\\
        \twentyitem
	    {TAF}
		{}
        {}
        {}
        {}
        {
        {\begin{itemize}
        \item SeleniumIDE, TestComplete
		\end{itemize}}
		}
		\\
        \twentyitem
	    {UML}
		{}
        {}
        {}
        {}
        {
        {\begin{itemize}
        \item MagicDraw, Bizagi, Mockups
		\end{itemize}}
		}
\end{twenty}
\end{document} 
