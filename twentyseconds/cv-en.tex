%%%%%%%%%%%%%%%%%%%%%%%%%%%%%%%%%%%%%%%%%
% Twenty Seconds Resume/CV
% LaTeX Template
% Version 1.0 (14/7/16)
%
% Original author:
% Carmine Spagnuolo (cspagnuolo@unisa.it) with major modifications by 
% Vel (vel@LaTeXTemplates.com) and Harsh (harsh.gadgil@gmail.com)
%
% License:
% The MIT License (see included LICENSE file)
%
%%%%%%%%%%%%%%%%%%%%%%%%%%%%%%%%%%%%%%%%%

%----------------------------------------------------------------------------------------
%	PACKAGES AND OTHER DOCUMENT CONFIGURATIONS
%----------------------------------------------------------------------------------------

\documentclass[letterpaper]{twentysecondcv} % a4paper for A4

% Command for printing skill overview bubbles
\newcommand\skills{ 
~
	\smartdiagram[bubble diagram]{
        \textbf{Software}\\\textbf{Engineer},
        \textbf{Unit}\\\textbf{testing},
        \textbf{~~~~~~~~Q\&A~~~~~~~~~},
        \textbf{~~~~~~Web~~~~~~}\\\textbf{~~development~~},
        \textbf{~~~~~OOP~~~~~}
    }
}

% Programming skill bars
\programming{{Vue.js $\textbullet$ Vuetify.js / 2.5}, {SQL$\textbullet$ HTML$\textbullet$ CSS$\textbullet$JavaScript / 3}, {C\# $\textbullet$ Java / 3.5}, {Python $\textbullet$ Django / 3}}

% Projects text
\education{
\textbf{Informatics Engineering} \\
Technological University of Havana
José Antonio Echeverría (CUJAE) \\
2010 - 2016 | Havana, Cuba

\textbf{Bachelor Diploma}  \\
IPUEC Manuel Ascunce Domenech \\
2007 - 2010 | Havana, Cuba
}

%----------------------------------------------------------------------------------------
%	 INFORMACIÓN PERSONAL
%----------------------------------------------------------------------------------------
% If you don't need one or more of the below, just remove the content leaving the command, e.g. \cvnumberphone{}

\cvname{MADELIN DIAZ} % Your name
\cvjobtitle{ Software Engineer } % Job
% title/career

\cvlinkedin{/in/madelin-díaz-ramos/}
\cvgithub{github.com/madelinkind}
\cvnumberphone{(+53) 535 38731} % Phone number
\cvsite{madelinkind.github.io} % Personal website
\cvmail{madelinkind@gmail.com} % Email address

%----------------------------------------------------------------------------------------

\begin{document}

\makeprofile % Print the sidebar
 
%----------------------------------------------------------------------------------------
%	 EXPERIENCE
%----------------------------------------------------------------------------------------

\section{Experience}

\begin{twenty} % Environment for a list with descriptions
\twentyitem
    	{09/16 - Now}
		{}
        {Quality Assurance Engineer}
        {\href{http://datys.cu/}{DATYS, Havana, Cuba}}
        {}
        {
        {\begin{itemize}
        \item Assimilation of the documentation associated with the company's solutions, ensuring that the software works in accordance with the requirements; work with their interfaces in the expected way, detecting defects early and preventing their propagation and arrival to the client (internal or external). To comply with the above, a series of activities mentioned below are carried out:
\item Design of the test plan to define the objectives of the system tests.
\item Define test cases based on functional, non-functional and technical requirements.
\item Integration test, define the Integration tests that will be carried out.
\item Run the test cases
\item Carry out the documentation of the evidence (evidence).
\item Record the incidents in the base to the defects found, as well as follow up to ensure proper correction.
\item Analyze and report the results of tests, statistics, etc.
        \end{itemize}}
        }
        
	%\twentyitem{<dates>}{<title>}{<location>}{<description>}
\end{twenty}

%----------------------------------------------------------------------------------------
%	 CERTIFICATION & TRAINING COURSES
%----------------------------------------------------------------------------------------
\section{Certifications and Training courses}

\begin{twenty} % Environment for a list with descriptions
\twentyitem
    	{2019}
		{}
        {Quality Model for the development of Computer Applications (MCDAI) (DATYS).}
        {}
        {}
        {
        {\begin{itemize}
        \item (Workshop) Exchange of experiences in the application of quality models, specifically the MCDAI.
		\end{itemize}}
        }
        \\
\twentyitem
    	{2018}
		{}
        {Quality evaluation of software and systems products (DATYS).}
        {}
        {}
        {
        {\begin{itemize}
        \item (Workshop) Execution of tests to computer applications, aimed at evaluating the quality characteristics of a software.
		\end{itemize}}
        }
        \\
\twentyitem
    	{2017}
		{}
        {BigData (DATYS).}
        {}
        {}
        {
        {\begin{itemize}
        \item Methods and tools to analyze information and facilitate business decision making.
		\end{itemize}}
        }
        \\
\twentyitem
    	{2016}
		{}
        {Advanced administration in GNU/Linux (DATYS).}
        {}
        {}
        {
        {\begin{itemize}
        \item Provide from GNU/Linux the necessary services to different environments of users and machines.
		\end{itemize}}
        }
        \\
\twentyitem
    	{2015}
		{}
        {Application Integration (CUJAE).}
        {}
        {}
        {
        {\begin{itemize}
        \item Integration of systems, applications and database.
		\end{itemize}}
        }
        \\
\twentyitem
    	{2015}
		{}
        {Analysis and design of algorithms (CUJAE).}
        {}
        {}
        {
        {\begin{itemize}
        \item Methodology for the solution of problems, using the computer through the design of algorithms.
		\end{itemize}}
        }
        \\
\twentyitem
    	{2015}
		{}
        {Data mining algorithms (CUJAE).}
        {}
        {}
        {
        {\begin{itemize}
        \item Introduction to the basic techniques of data mining.
		\end{itemize}}
        }
         \\
\twentyitem
    	{2014}
		{}
        {Design Sciences (CUJAE).}
        {}
        {}
        {
        {\begin{itemize}
        \item Fundamentals of design science.
		\end{itemize}}
        }

	%\twentyitem{<dates>}{<title>}{<location>}{<description>}
\end{twenty}
\pagebreak

\begin{twenty}
\twentyitem
    	{}
		{}
        {}
        {}
        {}
        {}
        \\
 \end{twenty} 
%----------------------------------------------------------------------------------------
%	 LANGUAGES
%----------------------------------------------------------------------------------------
\section{Languages}

\begin{twenty} % Environment for a list with descriptions
\twentyitem
    	{Spanish}
		{}
        {Native tongue}
        {}
        {}
        {}
        \\
	\twentyitem
    	{English}
		{}
        {Beginner}
        {}
        {}
        {}


	%\twentyitem{<dates>}{<title>}{<location>}{<description>}
\end{twenty}

%----------------------------------------------------------------------------------------
%	 	OTHER TECHNOLOGIES
%----------------------------------------------------------------------------------------  
\section{Technologies}
\begin{twenty}
	
	 \twentyitem
    	{Languages}
		{}
        {}
        {}
        {}
        {
        {\begin{itemize}
        \item C\#, Python, Java
		\end{itemize}}
        }
        \\
	\twentyitem
    	{IDEs}
		{}
        {}
        {}
        {}
        {
        {\begin{itemize}
        \item Microsoft Visual Studio, PyCharm
		\end{itemize}}
        }
        \\
        \twentyitem
    	{DVCS}
		{}
        {}
        {}
        {}
        {
        {\begin{itemize}
        \item Git (GitHub, GitLab), Team Foundation Server (TFS)
		\end{itemize}}
		}
		\\
        \twentyitem
	    {DBMS}
		{}
        {}
        {}
        {}
        {
        {\begin{itemize}
        \item PostgreSQL, SQLite.
		\end{itemize}}
		}
		\\
        \twentyitem
	    {CV}
		{}
        {}
        {}
        {}
        {
        {\begin{itemize}
        \item OpenCV, OpenCL.
		\end{itemize}}
		}
		\\
        \twentyitem
	    {Web}
		{}
        {}
        {}
        {}
        {
        {\begin{itemize}
        \item HTML, CSS, JavaScript, Vue.js, Vuetify.js. Web frameworks like Django.
		\end{itemize}}
		}
		\\
        \twentyitem
	    {Desktop}
		{}
        {}
        {}
        {}
        {
        {\begin{itemize}
        \item .NET, .NET Core.
		\end{itemize}}
		}
		\\
        \twentyitem
	    {OS}
		{}
        {}
        {}
        {}
        {
        {\begin{itemize}
        \item Microsoft Windows (XP, 7, 8, 10), GNU/Linux (Ubuntu)
		\end{itemize}}
		}
		\\
        \twentyitem
	    {TAF}
		{}
        {}
        {}
        {}
        {
        {\begin{itemize}
        \item SeleniumIDE, TestComplete
		\end{itemize}}
		}
		\\
        \twentyitem
	    {UML}
		{}
        {}
        {}
        {}
        {
        {\begin{itemize}
        \item MagicDraw, Bizagi, Mockups
		\end{itemize}}
		}
\end{twenty}
\end{document} 
